\documentclass[12pt]{article}
\usepackage[spanish]{babel}
\usepackage[utf8x]{inputenc}
\usepackage[a4paper,top=2cm,bottom=2cm,left=2.5cm,right=2.5cm,marginparwidth=1.75cm,headheight=28pt]{geometry}
\usepackage{enumerate}
\pagestyle{plain}

\title{Estatuto de la Sección estudiantil de ASME en el ITBA}
\author{Documento para ser revisado y eventualmente adoptado.}

\begin{document}
\maketitle

\subsection*{Nombre}
El nombre de nuestra organización es \textit{La Sección Estudiantil de la Asociación de Ingenieros Mecánicos Americana en el Instituto Tecnológico de Buenos Aires} (en adelante, \textbf{la Sección}). 
\subsection*{Misión}
La misión de la Sección es promover los objetivos de ASME, ayudar a los alumnos de la facultad en la carrera, y favorecer las conexiones y comunicaciones de los miembros con ingenieros dentro de la profesión y en la Asociación. 
%Dejo estas cosas en las que basé el nuevo artículo. El anterior era medio difuso, entonces busque a ver qué opinaba asme de todo esto y me parece que va al punto. Así como lo escribí, no es la mejor, pero creo que se puede trabajar con eso como guia. 

%Lo anterior era 'promover los La mision del captulo es facilitar/promover el conocimiento ingenieril mecanico y acercar el mundo laboral a los miembros del Captulo y del ITBA.'


% ASME Student Sections
% There are currently over 575 ASME student sections. ASME Student Sections are chartered at
% mechanical engineering colleges and universities around the world with the following purposes:
%  to promote the goals of the Society;
%  to complement the curriculum of the educational institution,
%  to offer mutual support in study, learning, and professionalism;
%  to provide personal connections and communications within the Society and the profession, and
%  to provide outreach and service to society in general

%
%The purposes of this Society are to:
%  Promote the art, science and practice of mechanical and
% multidisciplinary engineering and allied sciences to
% diverse communities throughout the world;
%  Encourage original research;
%  Foster engineering education;
%  Advance the standards of engineering;
%  Promote the exchange of information among engineers
% and others;
%  Broaden the usefulness of the engineering profession in
% cooperation with other engineering and technical
% societies; and
%  Promote a high level of ethical practice.

\section{Membresía}

\subsection{Condiciones} 
Se considera miembro de la Sección cualquier alumno que curse en el Instituto Tecnológico de Buenos Aires y que tenga 30 días de antigüedad en la Sociedad de Ingenieros Mecánicos Americano (ASME). 

\subsection{Derechos de los miembros}
\begin{enumerate}[a)]
\item Ser tratado con igualdad y respeto, sin diferenciar por género, etnia, credo, orientación sexual, condición social, afiliaciones políticas o mérito académico.
\item Ser debidamente informado de las actividades.
\item Ser invitado y participar de reuniones periódicas.
\item Expresar libremente su opinión y proponer actividades.
\end{enumerate}

\subsection{Obligaciones de los miembros}
\begin{enumerate}[a)]
\item Cumplir los reglamentos de la facultad y con los más altos estándares de ética durante su función como miembros de la Sección estudiantil.
\item Votar en las elecciones de los cargos.

\end{enumerate}

\section{Elecciones}

\begin{enumerate}[a)]
\item Todo cargo de la Sección será elegido mediante un sufragio universal, directo, secreto y obligatorio. El formato de la votación atenderá a estas características, sin importar la tecnología utilizada.
\item Las elecciones se celebrarán cada \emph{dos cuatrimestres}, dentro de los primeros 30 días de cursada.  
\item Son elegibles como candidatos los miembros de la Sección que tengan 100 dias de antigüedad y que sean estudiantes de la carrera de Ing. Mecánica.
\item Los candidatos tendrán acceso a los mails del electorado con el fin de divulgar sus propuestas.
\item Las fórmulas consisten en un candidato a Presidente junto con un candidato a Secretario. Una persona no podrá formar parte de dos fórmulas distintas.
\item El cargo de Vicepresidente corresponde al candidato a Presidente de la segunda fórmula más votada. 
\item El encargado fiscalizar la votación es el \emph{Faculty Advisor} sin perjuicio de los fiscales internos o de otras organizaciones que hayan sido invitadas. El Faculty Advisor será el árbitro para cualquier eventualidad que pudiera ocurrir, por ejemplo en caso de empate.  
\end{enumerate}

%CARGOS
\section{Cargos}
A continuación están listado los cargos que forman la mesa directiva.
\subsection{Presidente}
El presidente queda a cargo de cumplir las siguientes responsabilidades:
\begin{enumerate}[a)]
\item Liderar la Sección en la realización de actividades 
\item Representar a la Sección en asuntos interiores del ITBA y exteriores.
\item Organizar reuniones periódicas que sumen por lo menos 4 por cuatrimestre.
\item Nombrar cargos o equipos por escrito para la realización de eventos o cualquier otra actividad que fomente la misión de la Sección.
\end{enumerate}

\subsection{Vicepresidente}
El vicepresidente queda a cargo de cumplir las siguientes responsabilidades:
\begin{enumerate}[a)]
\item Administrar los distintos equipos. 
\item Tomar las responsabilidades del presidente en caso de su ausencia temporal o renuncia.
\end{enumerate}



\subsection{Secretario}
El secretario queda a cargo de cumplir las siguientes responsabilidades:
\begin{enumerate}[a)]
\item Resumir en detalle los temas tratados en reuniones y facilitar el acceso de estos resúmenes a los miembros. (Mantener el Libro de Actas)
\item Leer toda la correspondencia del E-mail oficial de la Sección y actuar responsablemente en el manejo del mismo.
\item Comunicarle al presidente todo mail que tenga importancia al cargo del mismo.
\item Divulgar noticias relevantes a la entre los miembros.
\item Fomentar el reclutamiento usando los instrumentos de comunicación oficiales de la Sección.
\end{enumerate}

\subsection{Tesorero}
El tesorero queda a cargo de cumplir las siguientes responsabilidades:
\begin{enumerate}[a)]
\item Administrar los bienes de la Sección.
\item Recibir y gestionar donaciones y compras, con la autorización del Presidente y el Faculty Advisor.
\item Mantener registros oficiales de todos los movimientos. 
\end{enumerate}
\section{Acerca del Estatuto}
\begin{enumerate}[a)]
\item El presente documento se podrá modificar cuando un miembro proponga una enmienda y ésta sea votada por 2/3 o más del total de los miembros (presentes y no presentes). La votación será bajo el arbitrio del Faculty Advisor
\item Se entregara una copia física y electrónica de este documento a cada miembro nuevo.
\end{enumerate}
\section{Enmiendas}
\end{document}


